\startabstractpage
{Performance and Power Characterization of Cellular Networks and Mobile Application Optimizations}
{Junxian Huang}{Chair: Z. Morley Mao}

Smartphones with cellular data access have become increasingly popular with the wide variety of mobile applications. However, the performance and power footprint of these mobile applications are not well-understood, and due to the unawareness of the cellular specific characteristics, many of these applications are causing inefficient radio resource and device energy usage. In this dissertation, we aim at providing a suite of systematic methodology and tools to better understand the performance and power characteristics of cellular networks (3G and the new LTE 4G networks) and the mobile applications relying upon, and to optimize the mobile application design based on this understanding.

We have built the \mobiperf tool to understand the characteristics of cellular networks. With this knowledge, we make detailed analysis on smartphone application performance via controlled experiments and via a large-scale data set from one major U.S. cellular carrier. To understand the power footprint of mobile applications, we have derived comprehensive power models for different network types and characterize radio energy usage of various smartphone applications via both controlled experiments and 7-month-long traces collected from 20 real users. Specifically, we characterize the radio and energy impact of the network traffic generated when the phone screen is off and propose the screen-aware traffic optimization. In addition to shedding light to the mobile application design throughout our characterization analysis, we further design and implement a real optimization system \NAMEFULL, which uses historical traffic features to make predictions and intelligently deallocate radio resource for improved radio and energy efficiency.


\oldstuff{

keywords:
Cellular Networks
Network Characterization
Smartphone Energy Model
LTE Networks
Mobile Application Optimization
TCP in Cellular Networks


some suggestions from Morley:

Combine I and II.  (II is very short).
The thesis statement in I is not quite concise and clear enough.
It needs to be a single sentence that captures problem statement and your solution/approach.
The texts starting with "However..." are too long.

I haven't the texts in detail, but it appears that there is very little transition across chapters.

The results in III and IV are a little dated, please clearly mark when these experiments were done.
(Please do so for the other results as well in other chapters.)



The transition between chapters is quite weak and it should refer back to the
thesis statement. 

There are five technical chapters of completed work, which is quite extensive, but not as coherent as it could be.  One way to make the dissertation easier to navigate is to provide a more clear roadmap in chapter 1, which currently summarizes each technical chapter.  It would be useful to broadly characterize these five pieces into:
1. mobile network characterization  (performance and RRC models)
2. mobile application characterization (protocol interaction)
    interplay between mobile app and network: energy implications
3. optimization to improve performance and energy.

By referring back to these three aspects, each of the five technical pieces can fit together better in the dissertation.

Chapter IX is written as if the author plans to continue to do some of the work.
It may be more appropriate to clarify the remaining challenges for others who are interested in following up on these topics.  It needs to clearly outline how these challenges may change as smartphones become more powerful, networks become faster and more energy efficient, and applications behavior also evolves.



Comments from Morley after defense

+ Add table of roadmap of results in intro so that it does not look like data dump.
+ What are the long term conclusions for cellular networks and what may change?

Please also take these comments into considerations besides the ones I gave you.

How persistent are some of the observations as networks and phones and applications evolve?  (maybe you want to emphasize the measurement methodology besides the measurement findings).

How are these observations different from Internet observations?

RadioProphet:  explain the principles more clearly.  need better comparison.
Idle time prediction is an old problem, how is it different from previous work?
What are the insights observed?
(During the defense, you seem to indicate that per application based prediction will not work, I agree with you, but per-application based prediction probably will be more accurate?  can you confirm that?

Slide 67: you need to also show the bad result: how much worse is your saving than FD).
Need to compare with less agressive FD (what is the common FD setting on today's phones?)

Introduction:
Needs to have a roadmap of different measurement results in all the chapters: when they are collected, what are the main findings.
Need to have a summary of the findings, key contributions, highlights of the important lessons learned.

}

\label{Abstract}
