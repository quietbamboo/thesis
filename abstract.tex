\startabstractpage
{Performance and Power Characterization of Cellular Networks and Mobile Application Optimizations}
{Junxian Huang}{Chair: Z. Morley Mao}

Smartphones with cellular data access have become increasingly popular with the wide variety of mobile applications. However, the performance and power footprint of these mobile applications are not well-understood, and due to the unawareness of the cellular specific characteristics, many of these applications are causing inefficient radio resource and device energy usage. In this dissertation, we aim at providing a suite of systematic methodology and tools to better understand the performance and power characteristics of cellular networks (3G and the new LTE 4G networks) and the mobile applications relying upon, and to optimize the mobile application design based on this understanding.

We have built the \mobiperf tool to understand the characteristics of cellular networks. With this knowledge, we also make microscopic analysis on smartphone application performance via controlled experiments and macroscopic analysis via the close examination of a large-scale data set from one major U.S. cellular carrier. To understand the power footprint of mobile applications, we have derived comprehensive power models for different network types and characterize radio energy usage of various smartphone applications via both controlled experiments and 7-month-long traces collected from 20 real users. Specifically, we characterize the radio and energy impact of the network traffic generated when the phone screen is off and propose the screen-aware traffic optimization. In additions to shedding light to the mobile application design throughout our characterization analysis, we further design and implement a real optimization system \NAMEFULL, which uses historical traffic features to make predictions and intelligently deallocate radio resource for improved radio and energy efficiency.


\oldstuff{
some suggestions from Morley:

Combine I and II.  (II is very short).
The thesis statement in I is not quite concise and clear enough.
It needs to be a single sentence that captures problem statement and your solution/approach.
The texts starting with "However..." are too long.

I haven't the texts in detail, but it appears that there is very little transition across chapters.

The results in III and IV are a little dated, please clearly mark when these experiments were done.
(Please do so for the other results as well in other chapters.)
}

\label{Abstract}
