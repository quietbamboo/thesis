\chapter{Anatomizing Smartphone Application Performance and Power Footprints} \label{chap:app}

Unlike traditional Internet-based applications, whose performance is mostly constrained by the wired network, network application performance on smartphones with limited physical resources also heavily depends on factors including hardware and software on the 
phone as well as the quality and load of wireless link. Understanding 
the application performance on smartphones is important for the 
purpose of assisting consumers in choosing carriers and phones and 
guiding application developers in designing intelligent software. 
Moreover, cellular network operators and smartphone hardware and 
software vendors can use this knowledge to optimize networks and 
phones for better end-user experiences. Similarly, content providers 
can leverage this knowledge to better customize content for mobile 
users. However, this task is quite challenging since the performance 
of network applications on smartphones is poorly understood before our study, due to a lack of a systematic approach for controlled experiments 
and comparative analysis. We believe this work fills this gap.

We focus on developing systematic methodology for measuring and
analyzing 3G network performance as well as smartphone application
performance. We make it relevant to end users by studying real
applications directly on the phone platforms. Our approach differs 
inherently from most previous work of using laptops equipped with 
3G data cards in three ways: (1) We measure the performance of 
applications rather than that of the low-level protocols. Prior 
work has shown that application performance often significantly 
deviates from protocol performance~\cite{Zhuang:A3:Mobicom2006}. 
We target the pervasive web browsing, streaming video, and VoIP
applications that most end-users care about; (2) We measure 
application performance on several common mobile devices. 
Application performance varies widely across devices due to 
differences in hardware and software, necessitating direct 
experimentation on smartphones instead of on laptops with wireless 
cards; (3) We study the application performance under real-world 
scenarios and quantify the performance of web browsing by 
evaluating commercial websites in addition to locally-constructed 
ones with replicated, real web content under our control. The 
latter setup helps dissect and analyze the individual factors 
that contribute to the overall web browsing performance.


In addition to shedding light on the overall application performance, we perform detailed analysis to identify and isolate
factors that impact user-perceived performance to help carriers,
phone vendors, content providers, and application developers gain 
insight. For example, for carriers, we infer various network-level 
problems, \eg high latency or high loss rate, which they can directly 
take action on. For phone vendors, we identify performance bottlenecks 
on the devices or issues associated with the content. These issues can
be resolved either independently or by cooperating with content 
providers. And for application developers, we evaluate factors such 
as the overhead of HTML rendering and Javascript execution given a 
particular software configuration.

We comprehensively study the 3G network and application performance
for all four major U.S. wireless carriers including AT\&T, Sprint,
Verizon, and T-Mobile. We choose popular devices including iPhone,
Android G2 from HTC, and Windows Mobile phones from Palm, HTC, and 
Samsung for carrying out experiments. Our results show that their
performance varies significantly across network applications. In 
fact, even for the same network application such as web browsing, 
certain types of phones consistently outperform others due to the 
differences in factors such as downloading behavior, customized 
contents, and page rendering. The application performance also 
heavily depends on properties of carriers including DNS lookup, 
RTT, and loss rate.

We summarize our main observations from extensive experimentation:

\begin{enumerate}%\itemsep=-0.9ex

\item The four carriers we studied demonstrate distinct 
characteristics in network performance in terms of throughput, RTT, 
retransmission rate, and time-of-day effect. For example, compared 
with T-Mobile and AT\&T's median of TCP retransmission rate of 
0\%, Sprint and Verizon have a higher median value of 0.7\%. 

\item TCP throughput, RTT, and retransmission rate vary widely even
for a single carrier in measurement taken at different times and 
locations, \eg downlink throughput ranges from 50 kbps to 4 Mbps for 
AT\&T, with the median value of about 1 Mbps.

\item The wireless delay in the 3G network dominates the whole
network path delay, \eg latency to the first pingable hop is around 
200 ms, which is close to the end-to-end Ping latency to landmark
servers distributed across the U.S.

\item Besides networks, devices heavily influence application 
performance. Given the same content and network condition, different 
devices exhibit vastly different webpage loading time, \eg the
page loading time of Samsung SCHi760 is consistently twice that of 
iPhone. 

\item Mobile devices can benefit from new content optimization
techniques like the data URL scheme, \eg~page loading time for 
GPhone can improve by 20\% in our experiments, despite 
its already good performance compared to other devices.
\end{enumerate}

\nsection{Methodology and Setup for Measuring Web Browsing Performance}
\label{sec:app.method}

In this section, we present our methodology and setup for measuring web browsing application performance over 3G networks. By analyzing the performance of web applications, we examine the effects of various factors on the overall application performance.

Unlike most previous works, we directly measure application 
performance on devices that consumers really use with 3G service 
provided by four major cellular carriers in the U.S., this helps us 
understand the client side factors and their impact on application 
performance. The novelty of our measurement methodology stems from 
our approach of approximately replicating the 3G network condition 
for controlled experiments using WiFi to enable reproducibility, 
and isolating the impact of each factor. These techniques are 
non-trivial given the complexity of mobile devices and network 
environment, and essential for eliminating interaction across factors. 

Web browsing is one of the most popular smartphone applications. The 
process of visiting a webpage can be quite complex given the dynamic 
nature of the content, often generated from Javascript, resulting in 
multiple concurrent TCP connections. Content can also be customized 
based on mobile device and carrier network.

Web browsing performance depends on various factors, \eg~DNS lookup 
time, TCP handshake time, TCP transfer time, Javascript execution 
time, and content size. To study the effect of these factors, we 
carefully design controlled experiments to modify a single factor 
at a time while keeping others the same. We first describe the metrics 
used to evaluate web browsing performance, followed by the controlled 
experiments to measure these metrics.

\nsubsection{Web Browsing Performance Metrics}
\label{sec:app.web_metrics}

\paragraph{Page loading time:} It is the time between the first DNS packet 
and the last data packet containing payload from the server during a 
page loading. It reflects the overall performance perceived by a user. 
Note that a browser needs to further parse and render a webpage after 
it is loaded. This additional processing cannot be fully included in 
page loading time due to a lack of visibility of the browser internals. 
Nonetheless, this is still a key indicator of user-perceived performance 
when loading a webpage. 

%For the rest of the paper, page loading time refers to page downloading time.

\paragraph{Javascript execution speed:} Many webpages contain 
Javascripts, and hence Javascript execution speed has significant
impact on page rendering time. 

%Page rendering time depends on the client execution speed.
%We use JavaScript execution time as a measure of client execution speed since a large portion 
%of webpage content consists of JavaScripts.


\paragraph{Page size:} The total number of unique bytes downloaded. 
It can be used to compute {\it average throughput} and to detect 
content variation and customization. 
%Page size is important to understand content customization
%for each platform. 
We found that in typical web browsing, even the same URL can have different 
page sizes when accessed from different platforms. We cope with this 
effect by taking snapshots of URLs and replicate their content on our 
local web server.

%\paragraph{Connection completion ratio} denotes the percentage of complete TCP
%connections. A TCP connection starts with a SYN packet and ends with a
%FIN packet, a RST packet, or 
%a timeout. We consider a connection to be complete if it ends with a FIN. 
%We discard a page download sample if its complete connection ratio is low 
%(\eg $<95\%$). 

%\paragraph{Loss rate \& RTT} are extracted from the trace for each TCP 
%connection. We aggregate the loss rate (denoted as $p$) and the 
%RTT of individual connections to produce an overall measure of network 
%performance of a page download. Given that TCP throughput can be modeled as 
%$\frac{MSS}{RTT\times\sqrt{p}}$ where $MSS$ is the maximum segment 
%size~\cite{Padhye:TCPModel:sigcomm2008}, 
%the download time of a connection of size $S$ can be estimated as 
%$\frac{S\times RTT\times \sqrt{p}}{MSS}$. Since the download time of a connection 
%is proportional to $S$, $RTT$, and $\sqrt{p}$, we use 
%$\frac{\sum(S_i \times RTT_i)}{\sum S_i}$ and 
%$(\frac{\sum{(S_i \times \sqrt{p_i})}}{\sum S_i})^2$ as the average RTT and loss 
%rate of a page download.
%\paragraph{Packet inter-arrival time}

\paragraph{Browser concurrency:} Most modern browsers support concurrent 
TCP connections to a single web domain. The maximum number of concurrent 
TCP connections to a domain varies across different browsers. Usually, 
higher concurrency enables better bandwidth utilization which in turn 
leads to shorter page loading time.


\paragraph{DNS lookup time:} A browser sometimes needs to look up the IP
address of a domain name before establishing a TCP connection with the
web server. Since the content of a webpage can be hosted in multiple 
domains, a browser may have to perform a DNS lookup for each domain. 

%DNS lookup time is important because DNS lookups can increase page loading time.
%Since DNS lookup requests are handled by local DNS (LDNS)
%servers, we also compute the average DNS lookup time as a measure of LDNS server 
%performance.

\paragraph{TCP handshake time:} Each TCP connection starts with a
three-way handshake during which no data is transferred. More TCP 
handshakes for a single page loading often lead to longer page loading 
time.

\paragraph{TCP idle time \& transfer time:} Given a TCP connection, 
an {\it idle period} is defined to be a period of at least $T$ second 
with no network activity. The remaining time periods within the 
connection are {\it transfer periods}. An idle period usually 
corresponds to the local processing delay or server processing 
delay. Given the limited CPU power and memory on smartphones, the 
TCP idle time is likely to be dominated by local processing delay, 
\eg between the receipt of a response and transmission of the next 
request, often caused by HTML rendering and JavaScript execution.

\nsubsection{Web Browsing Controlled Experiment Setup}

We select a list of 20 popular and representative URLs (based on Alexa Top 500 Global Sites~\cite{alexa}) including search engines, emails, online maps, social networking websites, \etc
For most of the URLs, we use their mobile versions, specially optimized for smartphone users, as opposed to desktop versions. To facilitate repeated experiments, we write a program to invoke browser to visit each URL in turn with an interval of 120 seconds. Such interval is expected to be large enough to complete the page download. We used \emph{Apache} 2.0 HTTP server for hosting the replicated websites. We collect packet traces on iOS and Android using {\em tcpdump} and on Windows Mobile phones using {\em netlog}. We verify that the CPU utilization caused by trace collection is under 5\%. All the experiments are repeated at least 10 times.

These websites are visited using smartphones via 3G networks. From the collected 
packet trace, we infer various metrics such as page loading time.
To study the effect of each factor influencing the web browsing
performance, we host static copies of these popular URLs on our 
local web server. The content is replicated to ensure that all the 
phones download the same content and all HTTP requests are sent to 
the local server. To control the network conditions, we uniformly 
use WiFi across all phones while varying one factor at a time. 
The WiFi link is lightly loaded and has stable throughput and RTT. 
To produce network conditions comparable to 3G, we artificially 
introduce delay and packet loss at our server. We study the impact 
of the following factors on web browsing performance:

\paragraph{Impact of network:} To study the effect of network
conditions on page loading time, we vary the RTT and loss rate 
on our server. To introduce artificial delay and loss, we ran a user-level program 
on the server. This program intercepts all packets destined to a 
particular IP address and injects delay and random packet loss. We 
controlled loss rate values from 0\% to 10\% and RTT values from 0~ms 
to 800~ms. These values cover the major range of loss rate and RTT 
values observed in 3G networks from our previous analysis. 

\paragraph{Impact of concurrency:} To study the effect of 
concurrency, we control the maximum number of concurrent TCP 
connections to a web domain on the server side by configuring the 
\emph{Apache} server with the help of the {\em mpm\_prefork\_module}.

Because a phone also limits the maximum number of concurrent connections per domain, we create a special webpage with 30 embedded objects in which each web object is hosted in a unique domain (a DNS alias) on the same web server. This effectively allows us to bypass the per-domain concurrency limit imposed by the phones. Note that this is necessary as we do not have the permission to directly modify the concurrency limit on the phone. For concurrency experiments, we use the RTT of 400 ms and the loss rate of 0\%. 

\paragraph{Impact of compression:} To study the tradeoff between 
network overhead and computation overhead, we configure our web
server into two modes, one uses compression, while the other does 
not. We compare the page loading time under these two modes.
Specifically, we use {\em SetOutputFilter}and {\em BrowserMatch} directives to specify whether compression is enabled for a specific type of browser, \ie compression is bypassed  by adding the following lines to the Apache config file:

\begin{lstlisting}
<Location />
	SetOutputFilter DEFLATE
	BrowserMatch Mozilla no-gzip
</Location>
\end{lstlisting}


We fix the loss rate at 0\% and vary the RTT from 0 ms to 800 ms. Our goal is to understand whether compression is beneficial under different network conditions.

\paragraph{Impact of Javascript execution speed:} To evaluate 
Javascript execution speed on different phones, we use a 
benchmark~\cite{sunspider} consisting of 26 different Javascripts. 
The benchmark is hosted on our web server and accessed by phones 
via WiFi so that the downloading time is negligibly small. We 
measure the total execution time of these Javascripts. 

\paragraph{Impact of the data URL scheme:} We also study the effect 
of the data URL scheme~\cite{rfc2397}, a recently-proposed mobile 
webpage design technique. We compare the time to load a webpage 
constructed using and without using the data URL scheme. 
Specifically, we construct a webpage with 20 images; 10 of them are of size 18KB and the remaining 10 are of size 0.241KB. We create two versions of this webpage, one with links to download the images and the other with the images embedded in the webpage itself. We could not carry out this experiment for Windows Mobile phones since IE does not support the data URL scheme in 2009.


\nsubsection{Analysis Methodology}

We next describe how to analyze the traces collected from controlled
experiments to compute the desired metrics. We calculate the page 
loading time of each URL as defined in Section~\ref{sec:app.web_metrics} 
and the average page loading time of all the selected URLs. To measure 
Javascript execution time, we modify the Javascripts to display their 
execution time when their execution finishes. We use the
{\em average concurrency} as a measure of browser concurrency. The
average concurrency of a page loading is calculated by dividing the 
total duration of all the TCP connections by the page loading time. 

For each TCP connection, TCP handshake time is calculated as the time
between the first \emph{SYN} and \emph{SYN-ACK} packets. TCP idle time 
is measured by scanning the connection for durations of more than 
\emph{T} seconds of no network activity. \emph{T} should be larger 
than the maximum RTT values. In our analysis, we choose \emph{T = 1} second, as we have seen that in cellular networks, even for 3G networks, RTT is smaller than 1 second in 99\% of the cases. Thus, if there
are no network activities for 1 second or more, the phone should 
be busy with some local processing, \eg rendering HTML pages or executing JavaScripts.
TCP transfer time is what remains for the connection excluding handshake and idle time. 
We also calculate the response time of all the DNS lookups $T_{dns}$. 

%given the difficulty of
%estimating the total time for the three-way handshake. 

Since each web browsing session often consists of multiple concurrent
TCP connections, to estimate the contribution of each factor to the 
overall performance, we logically serialize all DNS lookups and TCP 
connections. This is possible for mobile web browsing since no HTTP 
pipelining is observed on any phones. After serialization, we get a 
total time $T_{total}$ which is the sum of each connection's duration. 
Assuming the actual page loading time is $T^{*}_{total}$, the
normalized DNS lookup time $T^{*}_{dns}$ is calculated as
$T^{*}_{total} * T_{dns} / T_{total}$. This metric shows the 
overall weight of DNS lookup in the actual page loading time. The
normalized TCP handshake time, TCP idle time and TCP transfer time 
are calculated in a similar way. 
%We also get a temporary total time for
%DNS lookup ($T_{dns}$) by adding up the response time of each DNS
%lookup.
%For concurrent DNS lookups with the same request, only the response time of the first DNS
%request is counted because subsequent DNS lookups will not affect performance.	
	
	+ Web based application running time breakdown. Results from mobisys10
	

	+ Video/audio streaming applications (sigcomm submission, mobisys10)
%\nsubsection{Measuring streaming video performance}\label{sec:streaming}
%
%Streaming video is another popular application on smartphones. We measure 
%streaming video performance by playing a video on a phone. From the 
%collected packet trace, we calculate the downloading size of the video by 
%adding up the payloads for all the packets from the server to phone while 
%excluding the retransmitted packets. 
%
%\nsubsection{Measuring VoIP performance}\label{sec:voip}
%
%We use a popular VoIP application to call a wired computer from a 
%smartphone and play a piece of clip on both sides simulating 
%conversations. We collect packet trace at the phone side to calculate 
%throughput. Given that different volume may lead to different data size,
%throughout the experiment, we keep the volume to be the same.



	+ Screen status's impact on traffic pattern

	+ Power footprint of apps (mobisys12)
	
	