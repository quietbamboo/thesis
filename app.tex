\chapter{Anatomizing Smartphone Application Performance and Power Footprints} \label{chap:app}

\comment{smartphone application study (microscopic)}
	+ Motivation
	
	Unlike traditional Internet-based applications, whose performance is mostly constrained by the wired network, network application performance on smartphones with limited physical resources also heavily depends on factors including hardware and software on the 
phone as well as the quality and load of wireless link. Understanding 
the application performance on smartphones is important for the 
purpose of assisting consumers in choosing carriers and phones and 
guiding application developers in designing intelligent software. 
Moreover, cellular network operators and smartphone hardware and 
software vendors can use this knowledge to optimize networks and 
phones for better end-user experiences. Similarly, content providers 
can leverage this knowledge to better customize content for mobile 
users. However, this task is quite challenging since the performance 
of network applications on smartphones is poorly understood before our study, due to a lack of a systematic approach for controlled experiments 
and comparative analysis. We believe this work fills this gap.

We focus on developing systematic methodology for measuring and
analyzing 3G network performance as well as smartphone application
performance. We make it relevant to end users by studying real
applications directly on the phone platforms. Our approach differs 
inherently from most previous work of using laptops equipped with 
3G data cards in three ways: (1) We measure the performance of 
applications rather than that of the low-level protocols. Prior 
work has shown that application performance often significantly 
deviates from protocol performance~\cite{Zhuang:A3:Mobicom2006}. 
We target the pervasive web browsing, streaming video, and VoIP
applications that most end-users care about; (2) We measure 
application performance on several common mobile devices. 
Application performance varies widely across devices due to 
differences in hardware and software, necessitating direct 
experimentation on smartphones instead of on laptops with wireless 
cards; (3) We study the application performance under real-world 
scenarios and quantify the performance of web browsing by 
evaluating commercial websites in addition to locally-constructed 
ones with replicated, real web content under our control. The 
latter setup helps dissect and analyze the individual factors 
that contribute to the overall web browsing performance.

To overcome the limitation of a single vantage point for locally
conducted measurements, we design and deploy a cross-platform
measurement tool, called {\em 3GTest}, to measure network-level 
performance, using basic metrics such as throughput, round trip 
time (RTT), retransmission rate, \etc attracting more than 30,000 
users all over the world, providing a representative data set on 
the current 3G network performance. {\em 3GTest} enables us to 
carry out local experiments informed by realistic 3G network 
conditions across diverse locations and network carriers. As far 
as we know, \emph{3GTest} is the first such cross-platform tool 
available that comprehensively characterizes 3G network performance, 
and our data set is also unique in that regard.

In addition to shedding light on the overall application and network
performance, we perform detailed analysis to identify and isolate
factors that impact user-perceived performance to help carriers,
phone vendors, content providers, and application developers gain 
insight. For example, for carriers, we infer various network-level 
problems, \eg high latency or high loss rate, which they can directly 
take action on. For phone vendors, we identify performance bottlenecks 
on the devices or issues associated with the content. These issues can
be resolved either independently or by cooperating with content 
providers. And for application developers, we evaluate factors such 
as the overhead of HTML rendering and Javascript execution given a 
particular software configuration.

We comprehensively study the 3G network and application performance
for all four major U.S. wireless carriers including AT\&T, Sprint,
Verizon, and T-Mobile. We choose popular devices including iPhone,
Android G2 from HTC, and Windows Mobile phones from Palm, HTC, and 
Samsung for carrying out experiments. Our results show that their
performance varies significantly across network applications. In 
fact, even for the same network application such as web browsing, 
certain types of phones consistently outperform others due to the 
differences in factors such as downloading behavior, customized 
contents, and page rendering. The application performance also 
heavily depends on properties of carriers including DNS lookup, 
RTT, and loss rate.

We summarize our main observations from extensive experimentation:

\begin{enumerate}%\itemsep=-0.9ex

\item The four carriers we studied demonstrate distinct 
characteristics in network performance in terms of throughput, RTT, 
retransmission rate, and time-of-day effect. For example, compared 
with T-Mobile and AT\&T's median of TCP retransmission rate of 
0\%, Sprint and Verizon have a higher median value of 0.7\%. 

\item TCP throughput, RTT, and retransmission rate vary widely even
for a single carrier in measurement taken at different times and 
locations, \eg downlink throughput ranges from 50 kbps to 4 Mbps for 
AT\&T, with the median value of about 1 Mbps.

\item The wireless delay in the 3G network dominates the whole
network path delay, \eg latency to the first pingable hop is around 
200 ms, which is close to the end-to-end Ping latency to landmark
servers distributed across the U.S.

\item Besides networks, devices heavily influence application 
performance. Given the same content and network condition, different 
devices exhibit vastly different webpage loading time, \eg the
page loading time of Samsung SCHi760 is consistently twice that of 
iPhone. 

\item Mobile devices can benefit from new content optimization
techniques like the data URL scheme, \eg~page loading time for 
GPhone can improve by 20\% in our experiments, despite 
its already good performance compared to other devices.
\end{enumerate}



	
	
	
	
	+ Web based application running time breakdown.
	+ Video/audio streaming applications (sigcomm submission, mobisys10)
	+ Screen status's impact on traffic pattern
	+ Power footprint of apps (mobisys12)
