\chapter{MobiPerf: characterizing 3G/4G network performance} \label{chap:net}

Given the wide adoption of smartphone platforms, such as iOS and Android, there is a growing number of popular mobile applications designed for these platforms. For many of these applications, including web browser, email, VoIP, social networks, network access is required. Even for games that are often run locally, ranking systems and online peer matching systems are widely adopted which also requires network access, \eg Game Center for iOS. As a result, mobile data traffic volume is sky-rocketing. For example, AT\&T's mobile data volumes surged by a staggering 8,000\% from 2007 to 2010~\cite{att.overload}. Hence, it is critical to understand the {\em network performance} in cellular networks, and such understanding is a prerequisite to study smartphone application performance and optimizations. Smartphone customers want to know the cellular network performance in order to choose carriers and devices to use; mobile network operators care about cellular network performance to ensure the quality of service.

Systematically quantifying the cellular network performance is not straightforward. The challenges are multifold:
\begin{itemize}
\item It is not easy to reuse existing open-source network performance measurement tools due to smartphone operating system constraints. For example, iOS does not allow us to run a command-line program unless we jailbreak the device.
\item Conducting measurements from a single or a few vantage points is not sufficient for large cellular carriers. This is because the network condition and user load may vary across locations and without a reasonable number of sample users, the measurement results may not be representative. This forces us to abandon the idea to carry out all measurements ourselves, but instead to provide a tool for real smartphone users to use for network performance measurement.
\item As the measurement tool is intended to be run by real smartphone users, who do not necessarily have any computer science background, we have to design the tool in a way that is easy for these users to make network measurements.
\item The selection of the set of metrics for quantifying the cellular network performance is critical, as we want to collect sufficient information about the network performance within a period of user-tolerable time. Existing similar tools, such as Speedtest.net~\cite{speedtestnet} 
and FCC's broadband test~\cite{fccspeedtest}, only measure bandwidth and latency in cellular networks and ignore other important metrics such as DNS lookup time, \etc The measurement methodology also requires careful design, \eg bandwidth measurements in cellular networks need support from geo-distributed server nodes.
\end{itemize}


In the following of this chapter, we first discuss the design and implementation of \mobiperf, the tool to characterize cellular network performance in Section~\ref{sec:net.design}. Then we discuss the measurement results collected via \mobiperf and implications in Section~\ref{sec:net.result}.


\nsection{Design and Implication of MobiPerf}
\label{sec:net.design}

Inspired by previous work in the Internet, \eg~Netalyzr~\cite{netalyzr}, which collects measurement data from volunteers, we develop a measurement platform, \mobiperf~\footnote{\mobiperf~\cite{mobiperf} project was initiated in 2008 and the name of the measurement tool was \TT~\cite{mobiperf, mobisys.3gtest} at the time. With significant UI and methodology improvements, we changed the name to be \FT~\cite{4gtest} in 2011. In 2012, we further decided to change the tool name to be \mobiperf and made it open source~\cite{mobiperf.repo}. At the time this dissertation is being written, \mobiperf is under active development with joint efforts from University of Michigan, University of Washington, and M-Lab~\cite{mlab}.}, used by real users on their smartphones to build a comprehensive data set for cellular networks. The public deployment of \mobiperf overcomes the limitation of a single vantage point and short time duration for locally conducted measurements and provides a representative data set on cellular network performance in the real world.

\mobiperf covers a more comprehensive set of metrics than existing public network performance measurement tools available in iOS or Android, such as DNS lookup, Ping to the first hop, \etc. We next describe the metrics we use for evaluating network performance and how we compute them. To minimize the impact of the performance limiting factors in the Internet path, we leverage the M-Lab~\cite{mlab} support and make \mobiperf always choose the closest server node(s) for measurement. \mobiperf server suite is deployed to 46 M-Lab nodes across the world, covering North America, Europe, Asia, and Australia.  Each node has 4-core 2.00 GHz Intel Xeon CPU and our virtual slice has 4GB memory and 100Mbps Ethernet network access, which ensures that the network bottleneck is unlikely on the wired network path. Specifically, 23 nodes are within the U.S., spreading across major cities in different parts of the country. 

To characterize cellular network performance, we use TCP throughput, downlink RTT, retransmission rate, local DNS lookup time, TCP handshake time, and Ping latency to the first responsive IP hop as our metrics. TCP is of particular interest, since most network applications use TCP. An application session usually requires DNS lookup, and every TCP connection begins with a TCP handshake. Hence these two factors contribute heavily to user-perceived performance of many network applications. Ping latency to the first responsive hop provides an estimate of the latency of the wireless hop.

\paragraph{DNS lookup}
For the DNS experiment, \mobiperf sends DNS requests to resolve a list of selected popular domain names. We use Alexa~\cite{Alexa}top sites to select the top URLs and the list we use was downloaded in 2009. By tuning the size of the list and going through the list sequentially twice, we ensure that during the second lookup the names are highly likely cached at the local DNS (LDNS) server in carrier networks but not on the phone based on observed latencies. This is achievable since compared to the phone the LDNS server typically has a larger DNS cache. 


\paragraph{RTT and variation test}
LTE has significantly smaller latency compared with 3G~\cite{tr25.913}, hence the network distance between users and the measurement servers for the wired Internet path becomes less negligible. Given that the GPS coordinates for M-Lab nodes are known, in \mobiperf, a nearest M-Lab node is selected for a user based on the current GPS location if available, or the IP address otherwise, with the help of a IP address to GPS coordinates mapping~\cite{maxmind}. Such a mapping is sufficient for our purpose of finding coarse-grained location estimates for server selection.

To measure RTT and variation, \mobiperf repeatedly establishes a new TCP connection with the server and measures the delay between {\sf SYN} and {\sf SYN-ACK} packet. Both the median of these RTT measurements and the variation are reported to our central server. For some cellular ISPs, traffic may be redirected to a middlebox, which replies a {\sf SYN-ACK} packet to the client on behalf of the server. In this case, the measured RTT is between the client and the middlebox. However, this RTT is the user-perceived delay for TCP connection establishment and we think it is fine to use it for comparison purpose.

To measure TCP handshake to server nodes in diverse physical locations, \mobiperf sends TCP {\em connect} requests to different M-Lab nodes distributed across the U.S. To characterize Ping latency, our tool Pings \url{www.google.com} with increasing TTL values starting from 1 and records the IP address and the corresponding RTT. \mobiperf also Pings different M-Lab nodes to obtain the delay distribution to diverse Internet locations. 

\paragraph{TCP throughput}

Packet traces are collected at the server side 
to calculate TCP downlink and uplink throughput, RTT, and 
retransmission rate. 





Based on our previous experiences of developing , we design a new mobile network measurement tool for Android devices, called  , with higher accuracy and improved user experience. As a new feature, \FT allows users to switch among different network types, \ie 3G, WiFi and LTE. 



\paragraph{Throughput test}
Since single-threaded TCP measurement is more sensitive to packet loss and hence less accurate~\cite{sigcomm.broadband}, we use multi-threaded TCP measurement in \FT to estimate the {\em peak channel capacity}, \ie three nearest server nodes in M-Lab are selected for each user at runtime to start concurrent threads for throughput test.
Despite the shared nature of M-Lab nodes with other test tools, it is unlikely that all three selected nodes are overloaded in terms of CPU and network.

A throughput test lasts for 20 seconds, to balance across bandwidth usage, user waiting time and measurement accuracy. The initial 5 seconds are ignored empirically due to TCP slow start. The remaining 15 seconds are separated into 15 1-second bins. The average throughput for each bin is calculated and the median of all bins is the measured throughput. Compared with using average throughput, median more accurately estimates the steady peak throughput by reducing the impact of abnormal bins, \eg a very high bin due to initial buffering or a low bin due to temporary signal problem. Uplink and downlink tests share the same methodology.


We have made \mobiperf~\cite{mobiperf} publicly available, which allows us to characterize cellular network performance in multiple cellular carriers at diverse locations over an extended duration.


+ Metrics
+ Design and implementation of tools

\nsection{Understanding 3G/4G Network Performance}
\label{sec:net.result}
+ Results
+ Implications (compare across network types, compare over time)

talk about results from public deployment, also talk about the results in local controlled experiments via mobiperf, such as mobility test. This is another part of results for cellular network performance.
