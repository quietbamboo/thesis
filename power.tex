\chapter{Characterizing Radio Energy Usage of Smartphones in Cellular Networks} 
\label{chap:power}

Smartphones with cellular data access have become increasingly popular across the globe, with the wide deployment of 3G and emerging LTE~\cite{3gpp.lte} networks, and a plethora of applications of all kinds. Cellular networks are typically characterized by limited radio resources and significant device power consumption for network communications. The battery capacity of smartphones cannot be easily improved due to physical constraints in size and weight. Hence, battery life remains a key determinant of end-user experience. Given the limited radio resources in these networks and device battery capacity constraints, optimizing the usage of these resources is critical for cellular carriers and application developers. Specifically, radio power is reported to be 1/3 to 1/2 of the total device power~\cite{mobisys.aro}, and in this chapter, we first devise a systematic way to characterize smartphone radio energy usage given packet traces as input and then we discuss our analysis results with real user traces.

\nsection{Smartphone Power Model}
\label{sec:background.power}

\begin{figure}[t]
\centering
\IG{figures/mobisys12/trace_all.eps} \\
\ncaption{Power states of LTE}
\label{fig:power.trace.all}
\end{figure}

%So the plot of power traces in Figure 3 consists of dots of readings and may be >>594mW. During the DRX in RRC_IDLE, the actual power curve is something like a "n" shape curve (an increase period, a fluctuating top, and a decrease period). The average of the whole on period is 594mW, but the pikes inside the on period can be >>594mW. Those spikes have very minor effect in the energy calculation, but they are very obvious to see in the power trace figure.

Given the description of LTE state machine, we illustrate the power traces of an Android smartphone in a commercial LTE network based on local experiments described in \S\ref{sec:method.power}. We observe that network activities match the corresponding state transitions indicated by different power levels.

Figure~\ref{fig:power.trace.all} shows the power trace of uploading at the speed of 1Mpbs for 10 seconds. With screen off, the energy is mostly consumed by the radio interfaces, as the power level is less than 20mW before $t_1$. At $t_1$, the application sends a TCP {\sf SYN} packet triggering \RI to \RC promotion, and the application waits for $T_{pro}$ until starting data transfer at $t_2$. Between $t_2$ and $t_3$, depending on the instant data rate, the power level fluctuates. We notice the power level during fast data transfer is significantly higher than the base power in \RC, which motivates us to incorporate data rates into our LTE power model. After the data transfer completes at $t_3$, the device remains in \RC for a fixed tail time $T_{tail}$, until $t_4$, when the device goes back to \RI. The periodicity of DRX between $t_3$ and $t_4$ is not obvious due to limited sample rate.

%If there is any data sent or received between $t_3$ and $t_4$, the tail timer gets reset and a new tail starts.

\begin{figure}[h]
\centering
\IG{figures/mobisys12/trace_zoom.eps} \\
\ncaption{Zoom-in view in \RC}
\label{fig:power.trace.zoom}
\end{figure}

Figure~\ref{fig:power.trace.zoom} is a 125$\times$ zoom-in view of Figure~\ref{fig:power.trace.all}'s tail, which clearly illustrates DRX activity in \RC~mode. The device activates its receivers to listen to downlink control channel. If downlink traffic is found awaiting, the device goes into Continuous Reception model; otherwise, the device goes into a dormant state without continuously checking downlink control channel, though still remaining in \RC~mode. The spikes appearing every 40 ms in Figure~\ref{fig:power.trace.zoom} match the on period of DRX in \RC.

In summary, this section provides necessary background information and some initial motivating observations for the following discussions on the network and power characterization of LTE network.




\nsection{Power Model Construction}
\comment{\S5 of mobisys12}
\nsubsection{Power Measurement Methodology}

\nsection{User Trace Based Tradeoff Analysis}

\nsubsection{trace-driven network/power simulation methodology (validation)}
\S3.3

\nsubsection{\S6.1, 6.2, 6.3}
