\chapter{Characterizing Radio Energy Usage of Smartphones in Cellular Networks} 
\label{chap:power}

Smartphones with cellular data access have become increasingly popular across the globe, with the wide deployment of 3G and emerging LTE~\cite{3gpp.lte} networks, and a plethora of applications of all kinds. Cellular networks are typically characterized by limited radio resources and significant device power consumption for network communications. The battery capacity of smartphones cannot be easily improved due to physical constraints in size and weight. Hence, battery life remains a key determinant of end-user experience. Given the limited radio resources in these networks and device battery capacity constraints, optimizing the usage of these resources is critical for cellular carriers and application developers. Specifically, radio power is reported to be 1/3 to 1/2 of the total device power~\cite{mobisys.aro}, and in this chapter, we first devise a systematic way to characterize smartphone radio energy usage given packet traces as input and then we discuss our analysis results with real user traces.

\nsection{Smartphone Power Model}
\comment{RRC background should have been discussed already in background chapter}
\comment{\S2.2 of mobisys12}

\nsection{Power Model Construction}
\comment{\S5 of mobisys12}
\nsubsection{Power Measurement Methodology}

\nsection{User Trace Based Tradeoff Analysis}

\nsubsection{trace-driven network/power simulation methodology (validation)}
\S3.3

\nsubsection{\S6.1, 6.2, 6.3}
