\chapter{Characterizing Radio Energy Usage of Smartphones in Cellular Networks} 
\label{chap:power}

Smartphones with cellular data access have become increasingly popular across the globe, with the wide deployment of 3G and emerging LTE~\cite{3gpp.lte} networks, and a plethora of applications of all kinds. Cellular networks are typically characterized by limited radio resources and significant device power consumption for network communications. The battery capacity of smartphones cannot be easily improved due to physical constraints in size and weight. Hence, battery life remains a key determinant of end-user experience. Given the limited radio resources in these networks and device battery capacity constraints, optimizing the usage of these resources is critical for cellular carriers and application developers. Specifically, radio power is reported to be 1/3 to 1/2 of the total device power~\cite{mobisys.aro}, and in this chapter, we first devise a systematic way to characterize smartphone radio energy usage given packet traces as input and then we discuss our analysis results with real user traces.

\nsection{Power Measurement Methodology}

Similar to previous studies~\cite{imc.3g, codes.powertutor}, we use Monsoon power monitor~\cite{monsoon} as power input for our device measuring power traces at the same time. The power trace contains two fields, timestamp and average instant power, and the sampling rate is 5000Hz. The test device is an HTC phone with LTE data plan from a cellular ISP. It has 768 MB RAM memory and 1 GHz Qualcomm MSM8655 CPU, running Android 2.2.1. We remove the battery and connect $\oplus$ and $\ominus$ pins of the power monitor to device's $\oplus$ and $\ominus$ pins, respectively. By enabling $V_{out}$ with the voltage of 3.7V, the device boots properly and the power monitor records the total power traces consumed by the device. Another tricky part for our setup is that, the back cover of our test device, HTC Thunderbolt, must remain attached firmly, otherwise, if it is pried off the device, LTE session is terminated and 1xRTT session is started. This is because part of the LTE antenna circuit lies inside the back cover~\cite{thunderbolt}. In the end, we let the wires going out of the back cover through the hole of earplugs.

%In Section~\ref{sec:background.power},  Figure~\ref{fig:trace.all} and Figure~\ref{fig:trace.zoom} visualize one such power measurement in LTE network.
We share the same observation with previous study~\cite{codes.powertutor} that screen plays an important role in device power consumption, \ie with screen 100\% on, the UE idle power is 847.15mW compared with 11.36mW with screen off. For all measurements, we keep the test application running in the background with screen completely off to minimize power noise, unless UI interactions are required and screen should be kept on, \ie measuring power for browser. In this case, we subtract screen power from the total, with slightly increased noise. All experiments are repeated at least 5 times to reduce measurement error.

To measure state transition power levels, UE keeps a long-lived TCP connection with the server and packet traces are collected to make sure there is no background traffic. In order to trigger state promotions, we make the device idle for sufficient time, \eg 30 seconds, and then send a packet from server to client. UE remains idle afterwards and demotes to idle state in the end, and the power trace covers the full tail.

\nsection{Smartphone Power Model}

In Section~\ref{sec:bkg.rrc}, we have discussed the radio resource control mechanisms in cellular network. In fact, RRC state machine is the key factor for determining the UE power consumption for both 3G and LTE 4G networks.

In this section, we take LTE network as a sample and illustrate the power traces of our test Android smartphone in a commercial LTE network based on local experiments. We observe that network activities match the corresponding state transitions indicated by different power levels.

\begin{figure}[t]
\centering
\IG{figures/mobisys12/trace_all.eps} \\
\ncaption{Power states of LTE}
\label{fig:power.trace.all}
\end{figure}

%So the plot of power traces in Figure 3 consists of dots of readings and may be >>594mW. During the DRX in RRC_IDLE, the actual power curve is something like a "n" shape curve (an increase period, a fluctuating top, and a decrease period). The average of the whole on period is 594mW, but the pikes inside the on period can be >>594mW. Those spikes have very minor effect in the energy calculation, but they are very obvious to see in the power trace figure.

Figure~\ref{fig:power.trace.all} shows the power trace of uploading at the speed of 1Mpbs for 10 seconds. With screen off, the energy is mostly consumed by the radio interfaces, as the power level is less than 20mW before $t_1$. At $t_1$, the application sends a TCP {\sf SYN} packet triggering \RI to \RC promotion, and the application waits for $T_{pro}$ until starting data transfer at $t_2$. Between $t_2$ and $t_3$, depending on the instant data rate, the power level fluctuates. We notice the power level during fast data transfer is significantly higher than the base power in \RC, which motivates us to incorporate data rates into our LTE power model. After the data transfer completes at $t_3$, the device remains in \RC for a fixed tail time $T_{tail}$, until $t_4$, when the device goes back to \RI. The periodicity of DRX between $t_3$ and $t_4$ is not obvious due to limited sample rate.

%If there is any data sent or received between $t_3$ and $t_4$, the tail timer gets reset and a new tail starts.

\begin{figure}[h]
\centering
\IG{figures/mobisys12/trace_zoom.eps} \\
\ncaption{Zoom-in view in \RC}
\label{fig:power.trace.zoom}
\end{figure}

Figure~\ref{fig:power.trace.zoom} is a 125$\times$ zoom-in view of Figure~\ref{fig:power.trace.all}'s tail, which clearly illustrates DRX activity in \RC~mode. The device activates its receivers to listen to downlink control channel. If downlink traffic is found awaiting, the device goes into Continuous Reception model; otherwise, the device goes into a dormant state without continuously checking downlink control channel, though still remaining in \RC~mode. The spikes appearing every 40 ms in Figure~\ref{fig:power.trace.zoom} match the on period of DRX in \RC.



\nsection{Power Model Construction}
\comment{\S5 of mobisys12}


\nsection{User Trace Based Tradeoff Analysis}

\nsubsection{trace-driven network/power simulation methodology (validation)}
\S3.3

\nsubsection{\S6.1, 6.2, 6.3}
