\chapter{Conclusion and Future Work}
\label{chap:conc}

My dissertation is dedicated to understand and optimize the performance and energy footprint of smartphone applications, given their wide popularity. These issues are important for smartphone users, application developers, smartphone vendors, content providers and cellular network operators because improving the end-user experiences is their common interest. 

Given that smartphones and 3G/4G cellular networks are relatively new and such research topics are not well-studied, we have to solve many challenges and take some of the first steps in this area. For example, in order to characterize application performance, we need to understand the network characteristics of the underlying cellular networks and \mobiperf tool is built for this purpose. Also, in order to optimize the energy footprint of smartphone applications, we have devised a smartphone power model for 3G and LTE networks with real devices. The \mobiperf tool and the smartphone power model, along with the network and power characterization results, can potentially help other researchers in both industry and academia for their related research projects.

With controlled experiments separating individual factors on application performance, we are able to narrow down the bottleneck factors for web browsing applications and understand the effect of various performance optimization techniques. In addition, we study the interplay between TCP, application behavior and LTE network with a data set consisting of over 300,000 real LTE users and spot serious performance problems while identifying their root causes.

With the knowledge of smartphone power model and cellular radio resource management, we design and implement the \NAME system that makes intelligent decisions to dynamically release radio resources based on traffic pattern prediction. \NAME is implemented on Android incurring negligible CPU and energy overhead, while achieving radio energy savings and minimizing additional signaling overhead, significantly outperforming existing proposals. In addition, we also propose screen-aware optimization, which can better balance the key tradeoffs in cellular networks.

Throughout my research study, I make sure that the measurement results are {\em representative}, \eg via deploying smartphone applications to hundreds of thousand of real smartphone users and collect data directly from end devices. We also pay special attention to user privacy and only the anonymous measurement results that are beneficial for research are collected and all user identification information has been properly hashed. When designing systematic solutions for either measuring performance or optimizing energy footprint, we keep in mind that the best solution should be general, practical and long-lasting. Whenever possible, we implement and evaluate our system, \eg \NAME, on real devices, instead of simulations or synthetical analysis. We also pay special attention for the new LTE network, making sure we have thorough understanding of its network and power characteristics and our solutions are applicable for it, as it is believed to be a long-lasting cellular technology expected to exist for decades~\cite{3gpp.lte}.

\nsection{Future Research Agenda}

As future directions, I will continue to extend my previous works in the following aspects:
\begin{itemize}
\item \emph{Cellular Network Performance Characterization.} We will continue the development \emph{MobiPerf} and make it into a general mobile measurement platform that can benefit the entire research community. In addition, we will make use of the data set collected by \emph{MobiPerf} to make more analysis on cellular network performance and this data set~\cite{mobiperf.data} is also available for other researchers.
\item \emph{Energy Optimization for Cellular Networks.} We will improve the prediction algorithm of \NAME to consider TCP-level and application level context, which may further increase the prediction accuracy.
\item \emph{Effect of Network Protocol and Application Behavior on Performance for LTE Networks}. We will improve the design of the bandwidth estimation algorithm that we used to quantify the network utilization ratio of existing applications in the LTE networks. We will also explore the predictability of the available bandwidth in the LTE networks and study its implications and applications.
\end{itemize}
